%-------------------------------------------------------------------------------
%\subsubsection{Balancing Units}
%-------------------------------------------------------------------------------
Ensuring that units are balanced and correct requires care.
Take the description of the nonlinear system of ODEs that arises from the finite volume discretisation
\begin{equation}
    \label{eq:linsys_FV}
      V_i^{k+1} + \sum_{j\in\mathcal{N}_i} {\frac{\Delta t \alpha_{ij}}{\sigma_i} (V_i^{k+1}-V_j^{k+1})}
    = V_i^k - \frac{\Delta t}{c_m}(i_m^{k} - i_e).
\end{equation}
The choice of units for a parameter, e.g. $\mu m^2$ or $m^2$ for the area $\sigma_{ij}$, introduces a constant of proportionality wherever it is used ($10^{-12}$ in the case of $\mu m^2 \rightarrow m^2$).
Wherever terms are added in \eq{eq:linsys_FV} the units must be checked, and constants of proportionality balanced.

First, appropriate units for each of the parameters and variables are chosen in~\tbl{tbl:units}.
We try to use the same units as NEURON, except for the specific membrane capacitance $c_m$, for which $F\cdot m^{-2}$ is used in place of $nF\cdot mm^{-2}$.
In \eq{eq:linsys_FV} we choose units of $mV \equiv 10^{-3}V$ for each term because of the $V_i$ terms on either side of the equation.

\begin{table}[hp!]
\begin{tabular}{lllr}
    \hline
    term                      &   units                 &  normalized units                         & NEURON \\
    \hline
    $t$                       &   $ms$                  &  $10^{-3} \cdot s$                        & yes    \\
    $V$                       &   $mV$                  &  $10^{-3} \cdot V$                        & yes    \\
    $a,~\Delta x$             &   $\mu m$               &  $10^{-6} \cdot m$                        & yes    \\
    $\sigma_{i},~\sigma_{ij}$ &   $\mu m^2$             &  $10^{-12} \cdot m^2$                     & yes    \\
    $c_m$                     &   $F\cdot m^{-2}$       &  $s\cdot A\cdot V^{-1}\cdot m^{-2}$       & no     \\
    $r_L$                     &   $\Omega\cdot cm$ &    $  10^{-2} \cdot A^{-1}\cdot V\cdot m$      & yes    \\
    $\overline{g}$            &   $S\cdot cm^{-2}$      &  $10^{4} \cdot A\cdot V^{-1}\cdot m^{-2}$ & yes    \\
    $I_e$                     &   $nA$                  &  $10^{-9} \cdot A$                        & yes    \\
    \hline
\end{tabular}
\caption{The units chosen for parameters and variables in NEST MC. The NEURON column indicates whether the same units have been used as NEURON.}
\label{tbl:units}
\end{table}

%------------------------------------------
\subsubsection{current terms}
%------------------------------------------
Membrane current is calculated as follows $i_m = \overline{g}(E-V)$, with units
\begin{align}
    \unit{ i_m } &=  \unit{ \overline{g} } \unit{ V } \nonumber \\
                       &=  10^{4} \cdot A\cdot V^{-1}\cdot m^{-2} \cdot 10^{-3} \cdot V \nonumber \\
                       &=  10 \cdot A \cdot m^{-2}. \label{eq:im_unit}
\end{align}
The injected current $I_e$ has units $nA$, which has to be expressed in terms of current per unit area $i_e=I_e / \sigma_i$ with units
\begin{align}
    \unit{ i_e } &=  \unit{ I_e } \unit{ \sigma_i }^{-1} \nonumber \\
                       &=  10^{-9}\cdot A \cdot 10^{12} \cdot m^{-2} \nonumber \\
                       &=  10^{3} \cdot A \cdot m ^{-2}, \label{eq:ie_unit}
\end{align}
which must be scaled by $10^2$ to match $i_m$ in \eq{eq:im_unit}.

The units for the flux coefficent can be calculated as follows:
\begin{align}
    \unit{ \frac{\Delta t}{c_m} } &= 10^{-3} \cdot s \cdot s^{-1}\cdot A^{-1}\cdot V\cdot m^2 \nonumber \\
                                  &= 10^{-3} \cdot A^{-1} \cdot V\cdot m^2. \label{eq:dtcm_unit}
\end{align}
From \eq{eq:im_unit} and \eq{eq:dtcm_unit} that the units of the full current term are
\begin{align}
    \unit{ \frac{\Delta t}{c_m}\left(i_m - i_e\right) }
        &= 10^{-3} \cdot A^{-1} \cdot V\cdot m^2 \cdot 10 \cdot A \cdot m^{-2} \nonumber \\
        &= 10^{-2} \cdot V,
\end{align}
which must be scaled by $10$ to match the units of $mV\equiv10^{-3}V$.
%------------------------------------------
\subsubsection{flux terms}
%------------------------------------------
The coefficients in the linear system have the units
\begin{equation}
    \unit{ \frac{\Delta t\alpha_{ij}}{\sigma_i} }
    =
    \unit{ \frac{\Delta t \sigma_{ij} } {c_m r_L \Delta x_{ij} \sigma_i} }
    =
    \unit{ \frac{\Delta t } {c_m r_L \Delta x_{ij} } },
\end{equation}
where we we simplify by noting that $\unit{\sigma_{ij}}=\unit{\sigma_i}$.
The units of the term $c_m r_L$ on the denominator are calculated as follows
\begin{align}
    \unit{c_m r_L}
    &= s \cdot A \cdot V^{-1} \cdot m^{-2} \cdot 10^{-2} \cdot A^{-1} \cdot V \cdot m \nonumber \\
    &= 10^{-2} \cdot s \cdot m^{-1},
\end{align}
so the units of the denominator are
\begin{align}
    \unit{c_m r_L \Delta x_{ij}}
    &= 10^{-2} \cdot s \cdot m^{-1} \cdot 10^{-6} \cdot m \nonumber \\
    &= 10^{-8} \cdot s,
\end{align}
and hence
\begin{align}
    \unit{\frac{\Delta t } {c_m r_L \Delta x_{ij} }}
    &= 10^{8} \cdot s^{-1} \cdot 10^{-3} \cdot s \nonumber \\
    &= 10^{5}.
\end{align}

So, the terms with $\alpha_{ij}$ must be scaled by $10^5$ to match the units of $mV$.
%------------------------------------------
\subsubsection{discretization with scaling}
%------------------------------------------
Here is something that I wish the NEURON documentation had provided:
\begin{align}
&     V_i^{k+1} + \sum_{j\in\mathcal{N}_i} {10^5 \cdot \frac{\Delta t \alpha_{ij}}{\sigma_i} (V_i^{k+1}-V_j^{k+1})} \nonumber \\
&   = V_i^k - 10\cdot \frac{\Delta t}{c_m}(i_m^{k} - 10^2\cdot I_e/\sigma_i).
\end{align}
%------------------------------------------
\subsection{Supplementary Unit Information}
%------------------------------------------
Here is some information about units scraped from Wikipedia for convenience.

\begin{table*}[htp!]
    \begin{center}

    \begin{tabular}{llll}
        \hline
        quality & symbol & unit  & notes \\
        \hline
        energy     & $J$ & joule   $j$  & work to push 1 $N$ through 1 $m$ \\
        charge     & $q$ & coulomb $C$  & $6.25\cdot10^{18}$ electrons, $[A\cdot s]$ \\
        current    & $I$ & ampere  $A$  & $[C\cdot s^{-1}]$, $A$ is SI base unit\\
        voltage    & $V$ & volt    $V$  & potential work per unit charge \\
        resistance & $R$ & ohm $\Omega$ & recall Ohm's law $V=IR$ \\
        capacitance& $C$ & farad   $F$  & $C=\frac{q}{V}$, $[J\cdot C^{2}]$\\
        conductance& $g$ & siemens $S$  & $g=1/R$ i.e. inverse of resistance \\
        \hline
    \end{tabular}

    \vspace{20pt}

    \begin{tabular}{llll}
        \hline
        symbol & unit & equivalents & SI base \\
        \hline
        $J$    & $j$      &  $J\cdot s^{-1}$, $V\cdot A$ &
            $kg\cdot m^{2}\cdot s^{-2}$ \\

        $q$    & $C$      & $s\cdot A$ &
            $s\cdot A$ \\

        $I$    & $A$  & $C\cdot s^{-1}$ &
            $A$ \\

        $V$    & $V$      & $W\cdot A$ &
            $kg\cdot m^{2}\cdot s^{-3}\cdot A^{-1}$ \\

        $R$    & $\Omega$ & $V\cdot A^{-1}$ &
            $kg\cdot m^{2}\cdot s^{-3}\cdot A^{-2}$ \\

        $C$    & $F$      & $C\cdot V^{-1}$  &
            $kg^{-1}\cdot m^{-2}\cdot s^{4}\cdot A^{2}$ \\
        $g$    & $S$      & $A\cdot V^{-1}$  &
            $kg^{-1}\cdot m^{-2}\cdot s^3\cdot A^2$ \\
        \hline
    \end{tabular}

    \end{center}
    \caption{Symbols and quantities.}
\end{table*}

