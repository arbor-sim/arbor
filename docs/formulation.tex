%%%%%%%%%%%%%%%%%%%%%%%%%%%%%%%%%%%%%%%%%%%%%%%%%%%%%%%%%%%%%%%%%%%%%%%%%%%%%%%%
\subsection{Brief background for derivation of the cable equation}
%%%%%%%%%%%%%%%%%%%%%%%%%%%%%%%%%%%%%%%%%%%%%%%%%%%%%%%%%%%%%%%%%%%%%%%%%%%%%%%%
See \cite{lindsay_2004} for a detailed derivation of the cable equation, and extensions to the one-dimensional model that account for radial variation of potential.

The one-dimensional cable equation introduced later in equations~\eq{eq:cable} and~\eq{eq:cable_balance} is based on the following expression in three dimensions (based on Maxwell's equations adapted for neurological modelling)
\begin{equation}
    \nabla \cdot \vv{J} = 0,
\end{equation}
where $\vv{J}$ is current density (units $A/m^2$).
Current density is in turn defined in terms of electric field $\vv{E}$ (units $V/m$)
\begin{equation}
    \vv{J} = \sigma \vv{E},
\end{equation}
where $\sigma$ is the specific electrical conductivity of intra-cellular fluid (typically 3.3 $S/m$).

The derivation of the cable equation is based on two assumptions:
\begin{enumerate}
    \item that charge disperion is effectively instantaneous for the purposes of dendritic modelling.
    \item that diffusion of magnetic field is instant, i.e. it behaves quasi-statically in the sense that it is determined by the electric field through the Maxwell equations.
\end{enumerate}
Under these conditions, $\vv{E}$ is conservative, and as such can be expressed in terms of a potential field
\begin{equation}
    \vv{E} = \nabla \phi,
\end{equation}
where the extra/intra-cellular potential field $\phi$ has units $mV$.

The derivation of the one-dimensional conservation equation \eq{eq:cable_balance} is based on the assumption that the intra-cellular potential (i.e. inside the cell) does not vary radially.
That is, potential is a function of the axial distance $x$ alone
\begin{equation}
    \vv{E} = \nabla \phi = \pder{V}{x}.
\end{equation}
This is not strictly true, because a potential field that is a variable of $x$ and $t$ alone can't support the axial gradients required to drive the potential difference over the cell membrane.
I am still trying to get my head around the assumptions made in mapping a three-dimensional problem to a pseudo one-dimensional one.

%%%%%%%%%%%%%%%%%%%%%%%%%%%%%%%%%%%%%%%%%%%%%%%%%%%%%%%%%%%%%%%%%%%%%%%%%%%%%%%%
\subsection{The cable equation}
%%%%%%%%%%%%%%%%%%%%%%%%%%%%%%%%%%%%%%%%%%%%%%%%%%%%%%%%%%%%%%%%%%%%%%%%%%%%%%%%
The cable equation is a nonlinear parabolic PDE that can be written in the form
\begin{equation}
    \label{eq:cable}
    c_m \pder{V}{t} = \frac{1}{2\pi a r_{L}} \pder{}{x} \left( a^2 \pder{V}{x} \right) - i_m + i_e,
\end{equation}
where
\begin{itemize}
    \item $V$ is the potential relative to the ECM $[mV]$
    \item $a$ is the cable radius $(mm)$, and can vary with $x$
    \item $c_m$ is the {specific membrane capacitance}, approximately the same for all neurons $\approx 10~nF/mm^2$. Related to \emph{membrane capacitance} $C_m$ by the relationship $C_m=c_{m}A$, where $A$ is the surface area of the cell.
    \item $i_m$ is the membrane current $[A]$. The total contribution from ion and synaptic channels is expressed as a the product of current per unit area $i_m$ and the surface area.
    \item $i_e$ is the electrode current flowing into the cell, divided by surface area, i.e. $i_e=I_e/A$.
    \item $r_L$ is intracellular resistivity, typical value $1~k\Omega$
\end{itemize}

Note that the standard convention is followed, whereby membrane and synapse currents ($i_m$) are positive when outward, and electrod currents ($i_e$) are positive inward.

The PDE in (\ref{eq:cable}) is derived from the following mass balance expression for a cable segment $i$:
\begin{align}
    \int_{\Omega}{c_m \pder{V}{t} } \deriv{v} =
        & - \sum_{j\in\mathcal{N}_i} {\int_{\Gamma_{i,j}} \frac{1}{r_L}\pder{V}{x} n_{i,j} \deriv{s} } \nonumber \\
        & - \int_{\Gamma_{ext}} {(i_m - i_e)} \deriv{s}
    \label{eq:cable_balance}
\end{align}
where $\int_\Omega \cdot \deriv{v}$ is shorthand for the volume  integral over the segment $\Omega$, and $\int_\Gamma \cdot \deriv{s}$ is shorthand for the surface integral over the surface $\Gamma$.

The surface of the cable segment is sub-divided into the internal and external surfaces.
The external surface $\Gamma_{ext}$ is the cell membrane at the interface between the extra-cellular and intra-cellular regions.
The current, which is the conserved quantity in our conservation law, over the surface is composed of the synapse and ion channel contributions.
This is derived from a thin film approximation to the cell membrane, whereby the membrane is treated as an infinitesimally thin interface between the intra and extra cellular regions.

The internal surfaces are the interface between the cable segment and its neighbour segments which are denoted by the set $\mathcal{N}_i$.
Equation~\eq{eq:cable_balance} handles the general case where a cable might lie at a branch, and can be simplified for a one dimensional segment:
\begin{align}
    \int_{\Omega}{c_m \pder{V}{t} } \deriv{v} =
        \quad &   \int_{\Gamma_{\text{left}}}  \left( \frac{1}{r_L}\pder{V}{x} \right) \deriv{s} \nonumber \\
        -&  \int_{\Gamma_{\text{right}}} \left( \frac{1}{r_L}\pder{V}{x} \right) \deriv{s} \nonumber \\
        -&  \int_{\Gamma_{ext}} {(i_m - i_e)} \deriv{s}
    \label{eq:cable_balance_1D}
\end{align}

Note that some information is lost when going from a three-dimensional description of a neuron to a system of branching one-dimensional cable segments.
If the cell is represented by cylinders or frustrums\footnote{a frustrum is a truncated cone, where the truncation plane is parallel to the base of the cone.}, the three-dimensional values for volume and surface area at branch points can't be retrieved from the one-dimensional description.

\begin{figure}
    \begin{center}
        \includegraphics[width=0.5\textwidth]{./images/cable.pdf}
    \end{center}
    \caption{A single segment, or control volume, on an unbranching dendrite.}
    \label{fig:segment}
\end{figure}

%%%%%%%%%%%%%%%%%%%%%%%%%%%%%%%%%%%%%%%%%%%%%%%%%%%%%%%%%%%%%%%%%%%%%%%%%%%%%%%%
\subsection{Finite volume discretization}
%%%%%%%%%%%%%%%%%%%%%%%%%%%%%%%%%%%%%%%%%%%%%%%%%%%%%%%%%%%%%%%%%%%%%%%%%%%%%%%%
The finite volume method is a natural choice for the solution of the conservation law in~\eq{eq:cable_balance} and~\eq{eq:cable_balance_1D}.

\begin{itemize}
    \item   the $x_i$ are spaced uniformly with distance $x_{i+1}-x_{i} = \Delta x$
    \item   control volumes are formed by locating the boundaries between adjacent points at $(x_{i+1}+x_{i})/2$
    \item   this discretization differs from the finite difference method used in Neuron because the equation is explicitly solved for at the end of cable segments, and because the finite volume discretization is applied to all points. Neuron uses special algebraic \emph{zero area} formulation for nodes at branch points.
\end{itemize}

%-------------------------------------------------------------------------------
\subsubsection{Temporal derivative}
%-------------------------------------------------------------------------------
We proceed by defining the \emph{volume average} of a quantity $\varphi$ as follows:
\begin{equation}
    \bar{\varphi}_i = \frac{1}{v_i} \int_{\Omega_i}{\varphi}\deriv{v},
\end{equation}
where $\Omega_i$ is the control volume with the point $x_i$ at its centroid illustrated in \fig{fig:segment}, and $v_i$ is the volume of the control volume $\Omega_i$.
The integral one the left hand side of~\eq{eq:cable_balance} can be expressed in terms of the volume average of $V$
\begin{equation}
    \int_{\Omega}{c_m \pder{V}{t} } \deriv{v} = \frac{c_m}{\Delta_i} \pder{\bar{V}_i}{t}.
\end{equation}

In the FV formulation the voltage $V_i$ at the node $x_i$ is equal to the volume average, i.e. $V_i=\bar{V}_i$.
This is effectively treats voltage as a piecewise continuous funtion, with discontinuities at the boundary between adjacent segements.

%-------------------------------------------------------------------------------
\subsubsection{Intra-cellular flux}
%-------------------------------------------------------------------------------
The intracellular flux terms in~\eq{eq:cable_balance} are sum of the flux over interface between adjacent cable segments
\begin{equation}
    \sum_{j\in\mathcal{N}_i} {\int_{\Gamma_{i,j}}  \left( \frac{1}{r_L}\pder{V}{x} n_{i,j} \right) \deriv{s} }
\end{equation}

For the segment centred at $x_i$ with neighbour $x_j$
\begin{align}
    q_{i,j} &= \int_{\Gamma_{i,j}}  \left( \frac{1}{r_L}\pder{V}{x} n_{i,j} \right) \deriv{s} \nonumber \\
            &= \int_{\Gamma_{i,j}}  \frac{1}{r_L}\frac{V_i-V_j}{\Delta x_{i,j}} \deriv{s}
          \label{eq:q_ij_intermediate}
\end{align}
where:
\begin{itemize}
    \item $q_{i,j}$ is the \emph{flux} of charge over the interface $\Gamma_{i,j}$
    \item $\Delta x_{i,j}$ is the distance between $x_i$ and $x_j$, i.e. $\Delta x_{i,j}=|x_i-x_j|$.
\end{itemize}

The terms inside the integral in equation~\eq{eq:q_ij_intermediate} are constant everywhere on the surface $\Gamma_{i,j}$, so the integral becomes
\begin{align}
  q_{i,j} &= \int_{\Gamma_{i,j}}  \frac{1}{r_L}\frac{V_i-V_j}{\Delta x_{i,j}} \deriv{s} \nonumber \\
          &= \frac{1}{r_L}\frac{V_i-V_j}{\Delta x_{i,j}} \int_{\Gamma_{i,j}} 1 \deriv{s} \nonumber \\
          &= \frac{1}{r_L}\frac{V_i-V_j}{\Delta x_{i,j}} \sigma_{i,j} \nonumber \\
          &= \frac{\pi a_{i,j}^2}{r_L \Delta x_{i,j}} (V_i-V_j)
          \label{eq:q_ij}
\end{align}
where $\sigma_{i,j}=\pi a_{i,j}^2$ is the area of the surface $\Gamma_{i,j}$, which is a circle of radius $a_{r,j}$.

Some symmetries
\begin{itemize}
    \item $\sigma_{i,j}=\sigma_{j,i}$ : surface area of $\Gamma_{i,j}$
    \item $\Delta_{i,j}=\Delta_{j,i}$ : distance between $x_i$ and $x_j$
    \item $n_{i,j}=-n_{j,i}$ : surface ``norm''/orientation
    \item $q_{i,j}=n_{j,i}q_{i,j}=-q_{j,i}$ : charge flux over $\Gamma_{i,j}$
\end{itemize}

%-------------------------------------------------------------------------------
\subsubsection{Cell membrane flux}
%-------------------------------------------------------------------------------
The final term in~\eq{eq:cable_balance} with an integral is the cell membrane flux contribution
\begin{equation}
    q_{i}^{\text{m}} = \int_{\Gamma_{ext}} {(i_m - i_e)} \deriv{s},
\end{equation}
where the current $i_m$ is due to ion channel and synapses, and $i_e$ is any artificial electrode current.
The $i_m$ term is dependent on the potential difference over the cell membrane $V_i$.
The current terms are an average per unit area, therefore the total flux 
\begin{align}
    q_{i}^{\text{m}}
        = & (i_m(V_i) - i_e(x_i))\int_{\Gamma_{ext}} {1} \deriv{s} \nonumber \\
          & \sigma_i\cdot(i_m(V_i) - i_e(x_i)),
\end{align}
where $\sigma_i$ is the surface area the of the exterior of the cable segment, i.e. the surface corresponding to the cell membrane.

Each cable segment is modeled as a frustrum, with left and right radii $a_{i,\ell}$ and $a_{i,r}$.

%-------------------------------------------------------------------------------
\subsubsection{Putting it all together}
%-------------------------------------------------------------------------------
